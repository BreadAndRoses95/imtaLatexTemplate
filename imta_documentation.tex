%%%%%%%%%%%%%%%%%%%%%%%%%%%%%%%%%%%%%%%%%%%%%%%%%%%%%%%%%%%%%%%%%%%%%
% File Name:       main
%
% Description:     Documentation of the IMT Atlantique LaTeX Template.
%
% Note:            /
%
% Limitations:     /
%
% Errors:          None known
%
% Dependencies:    imta_core
%                  imta_extra
%
% Author:          A. Foucault - armand.foucault@telecom-bretagne.eu
% Contributors:    B. Porteboeuf - benoit.porteboeuf@telecom-bretagne.eu
%
% University :     IMT Atlantique, Brest (France)
%
% TeX Environment: TeXLive + pdfLaTeX
%%%%%%%%%%%%%%%%%%%%%%%%%%%%%%%%%%%%%%%%%%%%%%%%%%%%%%%%%%%%%%%%%%%%
% Revision List:
% Version    Author   Date         Changes
% 1.0        AF       15.10.2017   First Draft
% 1.0        BP       31.10.2017   Minor update
%%%%%%%%%%%%%%%%%%%%%%%%%%%%%%%%%%%%%%%%%%%%%%%%%%%%%%%%%%%%%%%%%%%%

\documentclass{article}


\usepackage{imta_core}  % IMTA Template's core (general layout and colors, compatible with standard TeX configurations)
\usepackage{imta_extra} % IMTA Template's more advanced functionalities (might not be compatible with standard TeX configurations)


\author{Armand FOUCAULT\\Benoît PORTEBŒUF}
\date{\today}
\title{\LaTeX{} report template}
\subtitle{Description of the \LaTeX{} template for IMT Atlantique reports}

\imtaSetIMTStyle



%%%%%%%%%%%%%%%%%%%%%%%%%%%%%%% 
%%%%%%%%%% BEGINNING %%%%%%%%%% 
\begin{document}

\imtaMaketitlepage

\tableofcontents
\newpage



%%%%%%%%%% PROJECT CREATION AND COMPILATION %%%%%%%%%% 

\section{Project creation and compilation}

\subsection{Project creation}

In order to use this template, create a new directory, and copy into it the following files:

\begin{itemize}
\item \imtaInlinecode{bash}{imta.tex}: this file declares all the data that composes the template.%
      It needs to be included to the main document, by calling \imtaInlinecode{latex}{%%%%%%%%%% PACKAGES %%%%%%%%%%

% Document geometry management
\usepackage[a4paper, margin=2cm, top=3cm]{geometry}

% Images management
\usepackage{graphicx}

% Floats management
\usepackage{float}

% Output font management
\usepackage[T1]{fontenc}

% Document encoding management
\usepackage[utf8]{inputenc}

% Include external pdf documents
% Used for generating the cover
\usepackage{pdfpages}

% Fancy headers and footers
\usepackage{fancyhdr}

% Fancy code listings
\usepackage{minted}

% Pictures drawing
\usepackage{tikz}

% Metadata access
\usepackage{titling}

% Arbitrary font size
\usepackage{anyfontsize}

% Fancy frames and boxes
\usepackage{mdframed}

% Section styling
\usepackage{sectsty}



%%%%%%%%%% COLORS %%%%%%%%%%

% The three colors of IMTA: green, light blue, and dark blue
\definecolor{imtaGreen}{RGB}{164, 210, 51}
\definecolor{imtaLightBlue}{RGB}{0, 184, 222}
\definecolor{imtaDarkBlue}{RGB}{12, 35, 64}
\definecolor{imtaGray}{RGB}{87, 87, 87}

% Colors related to code formatting
\definecolor{imtaCodeBackground}{RGB}{245, 245, 245}
\definecolor{imtaInlineCodeBackground}{RGB}{230, 230, 230}



%%%%%%%%%% GENERAL SETTINGS %%%%%%%%%%

\raggedbottom



%%%%%%%%%% PACKAGES SETTINGS %%%%%%%%%%

% minted
\setminted{linenos, breaklines}
\BeforeBeginEnvironment{minted}{%
    \vspace{0.5\baselineskip}
    \begin{mdframed}[backgroundcolor=imtaCodeBackground, innerleftmargin=5pt]%
}
\AfterEndEnvironment{minted}{%
    \end{mdframed}%
}
\renewcommand{\theFancyVerbLine}{\ttfamily \textcolor{gray!150}{\normalsize \oldstylenums{\arabic{FancyVerbLine}}}}

\setmintedinline{breaklines=false}


% fancyhdr
\pagestyle{fancy}
\fancyhead{}
\fancyfoot{}
\fancyhead[L]{\thetitle}
\fancyhead[R]{\theauthor}
\fancyfoot[C]{\thepage}

\fancypagestyle{imtaFirstpage}{%
    \fancyhf{}
    \renewcommand{\headrulewidth}{0pt}
}



%%%%%%%%%% COMMANDS %%%%%%%%%%

% \subtitle
% Command for the cover's subtitle
\newcommand{\subtitleValue}{}
\newcommand{\subtitle}[1]{%
    \renewcommand{\subtitleValue}{#1}
}

% \imtaInlinecode
% Typeset inline code
% First parameter is the language of the text to typeset
% Second parameter is the text to typeset
\newcommand{\imtaInlinecode}[2]{%
    \setlength{\fboxsep}{2pt}\colorbox{imtaInlineCodeBackground}{\mintinline{#1}{#2}}%
}

% \imtaQuestion
% Output "Question" followed by the current question number
% The \imtaQuestionReset command should be called after each new section or subsection,
% in order to resest the question counter
\newcounter{imtaQuestionCounter}
\newcommand{\imtaQuestion}{%
    \stepcounter{imtaQuestionCounter}%
    \subsection*{Question \arabic{imtaQuestionCounter}}%
}

% \imtaQuestionReset
% Reset the question counter
\newcommand{\imtaQuestionReset}{%
    \setcounter{\imtaQuestionCounter}{0}%
}

% \imtaMakeCover
% Output the IMTA title page
\newcommand{\imtaMaketitlepage}{%
    \thispagestyle{imtaFirstpage}
    \pagenumbering{gobble}
    \includepdf[pagecommand={%
        \begin{tikzpicture}[remember picture, overlay]
        \node[anchor=west, align=left] at (-0.5, -10.5) {%
            \large\fontfamily{phv}\selectfont\thedate \\
            \large\fontfamily{phv}\selectfont\theauthor \vspace{1.5cm}\\
            \textcolor{imtaGreen}{\fontsize{25}{40}\fontseries{b}\fontfamily{phv}\selectfont\thetitle} \vspace{.5cm}\\
            \textcolor{imtaGreen}{\Large\fontfamily{phv}\selectfont\subtitleValue}
        };
        \end{tikzpicture}
    }]{titlepage}
    \newpage
    \pagenumbering{arabic}
    \setcounter{page}{2}
}

% \imtaSetIMTStyle
% Set a styling that conforms to the official report template
\newcommand{\imtaSetIMTStyle}{%
    % Set global font to Helvetica
    \usepackage{helvet}
    \renewcommand{\familydefault}{\sfdefault}

    % Set heading style
    \sectionfont{\bf\LARGE\color{imtaGreen}}
    \subsectionfont{\bf\Large\color{imtaGray}}
    \subsubsectionfont{\bf\large\color{imtaGray}}
    \paragraphfont{\color{imtaGray}}
    \subparagraphfont{\color{imtaGray}}

    % Set header and footer style
    \pagestyle{fancy}
    \fancyhead{}
    \fancyfoot{}
    \fancyhead[L]{\nouppercase\leftmark}
    \fancyfoot[R]{\thepage}
    \fancypagestyle{imtaFirstpage}{%
        \fancyhf{}
        \renewcommand{\headrulewidth}{0pt}
    }
}



%%%%%%%%%% ENVIRONMENTS %%%%%%%%%% 

% imtaCode
% Typeset code listings
\newenvironment{imtaCode}[1]{
    \begin{minted}[frame=single, framesep=5pt, xleftmargin=\parindent, linenos, breaklines]{#1}
}{
    \end{minted}
}
}.
\item \imtaInlinecode{bash}{titlepage.pdf}: the title page of the document.%
      It is a blank version of the IMT Atlantique report template, over which an overlay title is written %
      by the \imtaInlinecode{latex}{\imtaMaketitlepage} command.
\end{itemize}

Your document should be a \imtaInlinecode{bash}{.tex} file, with the following skeleton:

% Report skeleton
\begin{imtaCode}{latex}
\documentclass{article}

%%%%%%%%%% PACKAGES %%%%%%%%%%

% Document geometry management
\usepackage[a4paper, margin=2cm, top=3cm]{geometry}

% Images management
\usepackage{graphicx}

% Floats management
\usepackage{float}

% Output font management
\usepackage[T1]{fontenc}

% Document encoding management
\usepackage[utf8]{inputenc}

% Include external pdf documents
% Used for generating the cover
\usepackage{pdfpages}

% Fancy headers and footers
\usepackage{fancyhdr}

% Fancy code listings
\usepackage{minted}

% Pictures drawing
\usepackage{tikz}

% Metadata access
\usepackage{titling}

% Arbitrary font size
\usepackage{anyfontsize}

% Fancy frames and boxes
\usepackage{mdframed}

% Section styling
\usepackage{sectsty}



%%%%%%%%%% COLORS %%%%%%%%%%

% The three colors of IMTA: green, light blue, and dark blue
\definecolor{imtaGreen}{RGB}{164, 210, 51}
\definecolor{imtaLightBlue}{RGB}{0, 184, 222}
\definecolor{imtaDarkBlue}{RGB}{12, 35, 64}
\definecolor{imtaGray}{RGB}{87, 87, 87}

% Colors related to code formatting
\definecolor{imtaCodeBackground}{RGB}{245, 245, 245}
\definecolor{imtaInlineCodeBackground}{RGB}{230, 230, 230}



%%%%%%%%%% GENERAL SETTINGS %%%%%%%%%%

\raggedbottom



%%%%%%%%%% PACKAGES SETTINGS %%%%%%%%%%

% minted
\setminted{linenos, breaklines}
\BeforeBeginEnvironment{minted}{%
    \vspace{0.5\baselineskip}
    \begin{mdframed}[backgroundcolor=imtaCodeBackground, innerleftmargin=5pt]%
}
\AfterEndEnvironment{minted}{%
    \end{mdframed}%
}
\renewcommand{\theFancyVerbLine}{\ttfamily \textcolor{gray!150}{\normalsize \oldstylenums{\arabic{FancyVerbLine}}}}

\setmintedinline{breaklines=false}


% fancyhdr
\pagestyle{fancy}
\fancyhead{}
\fancyfoot{}
\fancyhead[L]{\thetitle}
\fancyhead[R]{\theauthor}
\fancyfoot[C]{\thepage}

\fancypagestyle{imtaFirstpage}{%
    \fancyhf{}
    \renewcommand{\headrulewidth}{0pt}
}



%%%%%%%%%% COMMANDS %%%%%%%%%%

% \subtitle
% Command for the cover's subtitle
\newcommand{\subtitleValue}{}
\newcommand{\subtitle}[1]{%
    \renewcommand{\subtitleValue}{#1}
}

% \imtaInlinecode
% Typeset inline code
% First parameter is the language of the text to typeset
% Second parameter is the text to typeset
\newcommand{\imtaInlinecode}[2]{%
    \setlength{\fboxsep}{2pt}\colorbox{imtaInlineCodeBackground}{\mintinline{#1}{#2}}%
}

% \imtaQuestion
% Output "Question" followed by the current question number
% The \imtaQuestionReset command should be called after each new section or subsection,
% in order to resest the question counter
\newcounter{imtaQuestionCounter}
\newcommand{\imtaQuestion}{%
    \stepcounter{imtaQuestionCounter}%
    \subsection*{Question \arabic{imtaQuestionCounter}}%
}

% \imtaQuestionReset
% Reset the question counter
\newcommand{\imtaQuestionReset}{%
    \setcounter{\imtaQuestionCounter}{0}%
}

% \imtaMakeCover
% Output the IMTA title page
\newcommand{\imtaMaketitlepage}{%
    \thispagestyle{imtaFirstpage}
    \pagenumbering{gobble}
    \includepdf[pagecommand={%
        \begin{tikzpicture}[remember picture, overlay]
        \node[anchor=west, align=left] at (-0.5, -10.5) {%
            \large\fontfamily{phv}\selectfont\thedate \\
            \large\fontfamily{phv}\selectfont\theauthor \vspace{1.5cm}\\
            \textcolor{imtaGreen}{\fontsize{25}{40}\fontseries{b}\fontfamily{phv}\selectfont\thetitle} \vspace{.5cm}\\
            \textcolor{imtaGreen}{\Large\fontfamily{phv}\selectfont\subtitleValue}
        };
        \end{tikzpicture}
    }]{titlepage}
    \newpage
    \pagenumbering{arabic}
    \setcounter{page}{2}
}

% \imtaSetIMTStyle
% Set a styling that conforms to the official report template
\newcommand{\imtaSetIMTStyle}{%
    % Set global font to Helvetica
    \usepackage{helvet}
    \renewcommand{\familydefault}{\sfdefault}

    % Set heading style
    \sectionfont{\bf\LARGE\color{imtaGreen}}
    \subsectionfont{\bf\Large\color{imtaGray}}
    \subsubsectionfont{\bf\large\color{imtaGray}}
    \paragraphfont{\color{imtaGray}}
    \subparagraphfont{\color{imtaGray}}

    % Set header and footer style
    \pagestyle{fancy}
    \fancyhead{}
    \fancyfoot{}
    \fancyhead[L]{\nouppercase\leftmark}
    \fancyfoot[R]{\thepage}
    \fancypagestyle{imtaFirstpage}{%
        \fancyhf{}
        \renewcommand{\headrulewidth}{0pt}
    }
}



%%%%%%%%%% ENVIRONMENTS %%%%%%%%%% 

% imtaCode
% Typeset code listings
\newenvironment{imtaCode}[1]{
    \begin{minted}[frame=single, framesep=5pt, xleftmargin=\parindent, linenos, breaklines]{#1}
}{
    \end{minted}
}


\author{Author name}
\date{Writing date}
\title{Document name}
\subtitle{Short description or subtitle}

\begin{document}

\imtaMaketitlepage

\section{First section}

...

\end{document}

\end{imtaCode}
% Skeleton end


\subsection{Compilation}

This template is intended to be compiled with \imtaInlinecode{bash}{pdflatex}.
Furthermore, it makes use of the \imtaInlinecode{bash}{minted} package.
As a consequence, the compiler needs to be passed the \imtaInlinecode{bash}{-shell-escape} flag.
In addition, as usual when wishing a table of contents, the main document should be compiled twice, so as to make sure that the references refer to the right labels.
If your main document is called \imtaInlinecode{bash}{main.tex}, use the following command to compile it:

\begin{imtaConsole}
$ pdflatex -shell-escape main.tex
\end{imtaConsole}



%%%%%%%%%% PACKAGES %%%%%%%%%% 

\section{Packages}

This template uses a number of packages, with specific options.
Besides, some packages are further configured, through specific commands.
These can be found in the source code itself, at the \imtaInlinecode{latex}{PACKAGES SETTINGS} section.
The following is an abstract from the \imtaInlinecode{latex}{PACKAGES} section of the \imtaInlinecode{bash}{imta.tex} file.

\begin{imtaCode}{latex}

\usepackage[a4paper, margin=2cm, top=3cm]{geometry}
\usepackage{graphicx}
\usepackage{float}
\usepackage[T1]{fontenc}
\usepackage[utf8]{inputenc}
\usepackage{pdfpages}
\usepackage{fancyhdr}
\usepackage{minted}
\usepackage{tikz}
\usepackage{titling}
\usepackage{anyfontsize}
\usepackage{mdframed}

\end{imtaCode}

\subsection{\texttt{anyfontsize}}

The \imtaInlinecode{latex}{anyfontsize} package allows picking an arbitrary size for a local font.
It provides the \imtaInlinecode{latex}{\fontsize} command, used for generating the title page inside of a \imtaInlinecode{latex}{tikzpicture} environment.

\subsection{\texttt{fancyhdr}}

The \imtaInlinecode{latex}{fancyhdr} package lets define custom headers and footers.
For instance, the IMT Atlantique header and footer style is defined as follows (inside of the \imtaInlinecode{latex}{\imtaSetIMTStyle} command):

\begin{imtaCode}{latex}
\pagestyle{fancy}                       % Select the fancy style provided by fancyhdr

\fancyhead{}                            % Clear the current header style
\fancyfoot{}                            % Clear the current footer style

\fancyhead[L]{\nouppercase\leftmark}    % Define the content of the header:
                                        %     the current section title, on the left
\fancyfoot[R]{\thepage}                 % Define the content of the footer:
                                        %     the current page number, on the right

\fancypagestyle{imtaFirstpage}{%        % Define the style for the first page
    \fancyhf{}                          % Clear the current style
    \renewcommand{\headrulewidth}{0pt}  % Clear the horizontal rule under the header
}
\end{imtaCode}

\subsection{\texttt{float}}
\subsection{\texttt{fontenc}}
\subsection{\texttt{geometry}}
\subsection{\texttt{graphicx}}
\subsection{\texttt{inputenc}}
\subsection{\texttt{mdframed}}
\subsection{\texttt{minted}}
\subsection{\texttt{pdfpages}}
\subsection{\texttt{tikz}}
\subsection{\texttt{titling}}



%%%%%%%%%% COMMANDS %%%%%%%%%% 

\section{Commands}

This template provides a handful of new commands.

\subsection{Generic commands}

\subsubsection{Metadata commands}

This template defines a \imtaInlinecode{latex}{\subtitle} macro, that receives the subtitle of the document.
The latter will be displayed on the title page.
The purpose of this macro is to provide a consistent way of defining a subtitle, with regard to the \imtaInlinecode{latex}{\title}, %
\imtaInlinecode{latex}{\author}, and \imtaInlinecode{latex}{\date} standard macros.
It takes a single parameter, that is the subtitle to display.

\subsection{\texttt{imta} commands}

\subsubsection{Typeset inline code with \texttt{imtaInlinecode}}

\subsubsection{Output the title page with \texttt{imtaMaketitlepage}}

\subsubsection{Answer questions with \texttt{imtaQuestion} and \texttt{imtaQuestionReset}}

The \imtaInlinecode{latex}{\imtaQuestion} command outputs and formats a question counter.
It's meant to be used in reports for assignment with questions.
The counter should be reset with the \imtaInlinecode{latex}{\imtaQuestionReset}.
The output is as follows:

\imtaQuestion
Answer to first question

\imtaQuestion
Answer to second question

\imtaQuestionReset

\imtaQuestion
Answer to first question of the second section

\vspace{\baselineskip}
And the corresponding code:

\begin{minted}{latex}
\imtaQuestion
Answer to the first question

\imtaQuestion
Answer to the second question

\imtaQuestionReset

\imtaQuestion
Answer to the first question of the second section
\end{minted}



%%%%%%%%%% ENVIRONMENTS %%%%%%%%%% 

\section{Environments}

\subsection{Generic environments}

\subsubsection{Typeset code listings with \texttt{imtaCode}}

\begin{imtaCode}{cpp}
int a = 5;
\end{imtaCode}

\subsection{\texttt{imta} environments}

% \subsubsection{Typeset console sessions with \texttt{minted} and the \texttt{imtaConsole} style}
% 
% The \imtaInlinecode{latex}{imtaConsole} style for \imtaInlinecode{latex}{minted} provides a way to format console sessions.
% It outputs raw text in texttyper font, in a framed box.
% Here is an output example:
% 
% \begin{minted}{imtaConsole}
%     user@host:~$ mkdir newDirectory
%     user@host:~$ cd newDirectory
%     user@host:~$ touch newFile
% \end{minted}



%%%%%%%%%% STYLING %%%%%%%%%% 

\section{IMT Atlantique styling}

The official IMT Atlantique styling is not really \LaTeX-ish, and takes the decision to use a sans-serif font for body text.
Therefore, I chose to use the default \LaTeX{} font settings, which look much more professional.
Of course, this style does not suit the official report style.
Thus, I decided to provide a command that enables that official style.

The main aspects of the official style are:

\begin{itemize}
\item Use of the Helvetica font for the body;
\item Section titles in green (\imtaInlinecode{latex}{\imtaGreen}) and other heading titles in gray (\imtaInlinecode{latex}{\imtaGray});
\item Section title in the header;
\item Page number at the right corner of the footer.
\end{itemize}

For comparison, the default style of the template is:

\begin{itemize}
\item Use of the default Computer Modern font for the body;
\item Default style for headings: all in black;
\item Document title at the left corner and author's name at the right corner of the header;
\item Page number at the center of the footer.
\end{itemize}

The official IMT Atlantique style can be toggled with the \imtaInlinecode{latex}{\imtaSetIMTStyle} command.
Since it makes use of the \imtaInlinecode{latex}{\usepackage} macro, it needs to be called in the preamble.
No way is provided to disable later in the document the official style.
As a consequence, you cannot have half of the document with the official style, and the other half in the default style.


\imtaMakeCover


\end{document}
%%%%%%%%%% END %%%%%%%%%% 
%%%%%%%%%%%%%%%%%%%%%%%%% 
