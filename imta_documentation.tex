\documentclass{report}

\usepackage{imta_core}
\usepackage{imta_extra}

\usepackage[english]{babel}

\author{FOUCAULT Armand\\PORTEBOEUF Benoît}
%\imtaAuthorShort{A. Foucault \& B. Porteboeuf}
\date{December 2017}
\title{Institut Mines-Télécom Atlantique - \LaTeX{} report template}
\subtitle{Documentation for the \texttt{imta} package}

\imtaSetIMTStyle

%%%%%%%%%%%%%%%%%%%%%%%%%%%%%%% 
%%%%%%%%%% BEGINNING %%%%%%%%%% 
\begin{document}
	
\imtaMaketitlepage

\tableofcontents

\newpage

\chapter{Core features: \texttt{imta\_core}}
 
\section{Sectioning}

\subsection{\texttt{imtaQuestion} and \texttt{imtaQuestionReset}}
The \imtaInlinecode{latex}{\imtaQuestion} command outputs and formats a question counter.
It's meant to be used in reports for assignment with questions.
The counter should be reset with the \imtaInlinecode{latex}{\imtaQuestionReset}.
This couple of commands is meant to be used for sectioning when the assignment does not use a more explicit titling.

\subsection{\texttt{subsubsubsection}}
The \imtaInlinecode{bash}{imta_core} package offers a one-level-deeper section than the usual deepest \imtaInlinecode{latex}{\subsubsection}.
This provides an alternative to the usual \imtaInlinecode{latex}{\paragraph}.

\subsection{\texttt{chapter}}

\section{Document styling}

\subsection{IMT Atlantique styling}
\subsubsection{Colors}
The core package defines four colors, including the three colors of the IMT Atlantique, and a uniform and arbitrary gray.
These are defined as follows:

\begin{imtaCode}{latex}
\definecolor{imtaGreen}{RGB}{164, 210, 51}
\definecolor{imtaLightBlue}{RGB}{0, 184, 222}
\definecolor{imtaDarkBlue}{RGB}{12, 35, 64}
\definecolor{imtaGray}{RGB}{87, 87, 87}
\end{imtaCode}

Here are samples of these colors, with text in both black and white for previsualising the contrast.

\begin{figure}[H]
    \centering
    \resizebox{10cm}{!}{%
    \begin{tikzpicture}
        \fill[color=imtaGray] (0, 9) rectangle (6, 11);
        \fill[color=imtaGray] (7, 9) rectangle (13, 11);
        \node at (3, 10) {\LARGE \bf imtaGray};
        \node[white] at (10, 10) {\LARGE \bf imtaGray};

        \fill[color=imtaDarkBlue] (0, 6) rectangle (6, 8);
        \fill[color=imtaDarkBlue] (7, 6) rectangle (13, 8);
        \node at (3, 7) {\LARGE \bf imtaDarkBlue};
        \node[white] at (10, 7) {\LARGE \bf imtaDarkBlue};

        \fill[color=imtaLightBlue] (0, 3) rectangle (6, 5);
        \fill[color=imtaLightBlue] (7, 3) rectangle (13, 5);
        \node at (3, 4) {\LARGE \bf imtaLightBlue};
        \node[white] at (10, 4) {\LARGE \bf imtaLightBlue};

        \fill[color=imtaGreen] (0, 0) rectangle (6, 2);
        \fill[color=imtaGreen] (7, 0) rectangle (13, 2);
        \node at (3, 1) {\LARGE \bf imtaGreen};
        \node[white] at (10, 1) {\LARGE \bf imtaGreen};
    \end{tikzpicture}
    }
    \caption{Samples of the IMT Atlantique colors}
    \label{fig:imtaColors}
\end{figure}

\subsubsection{\texttt{imtaSetIMTStyle}}
The official IMT Atlantique styling is not really \LaTeX-ish, and takes the decision to use a sans-serif font for body text.
Therefore, the default style uses the default \LaTeX{} font settings.
However, it is possible to enable a more IMT Atlantique-compliant styling, by calling the \imtaInlinecode{latex}{\imtaSetIMTStyle} command in the preamble.

The main aspects of the official style are:

\begin{itemize}
    \item Use of the Helvetica font for the body;
    \item Section titles in green (\imtaInlinecode{latex}{\imtaGreen}) and other heading titles in gray (\imtaInlinecode{latex}{\imtaGray});
    \item Section title in the header;
    \item Page number at the right corner of the footer.
\end{itemize}

For comparison, the default style of the template is:

\begin{itemize}
    \item Use of the default Computer Modern font for the body;
    \item Default style for headings: all in black;
    \item Document title at the left corner and author's name at the right corner of the header;
    \item Page number at the center of the footer.
\end{itemize}

\subsubsection{The IMT Atlantique logo with \texttt{imtaLogo} and \texttt{imtaLogoTikz}}
The IMT Atlantique logo can be output at the desired width with the \imtaInlinecode{latex}{\imtaLogo} command.
The latter includes an external pdf document, \imtaInlinecode{bash}{imta_logo.pdf}, that contains the official logo.
On the other hand, the \imtaInlinecode{latex}{\imtaLogoTikz} draws an approximation of the logo with the \imtaInlinecode{latex}{tikz} package.
The following is a comparison of both commands.

\begin{figure}[H]
    \centering
    \imtaLogo{5cm}
    \imtaLogoTikz{5cm}
    \caption{Comparison between \imtaInlinecode{latex}{imtaLogo} and \imtaInlinecode{latex}{imtaLogoTikz}}
    \label{fig:imtaLogo}
\end{figure}


\subsubsection{\texttt{imtaMaketitlepage}}
This command outputs a title page with the names of the authors, the date of writing, and the title of the document, along with the subtitle.
For this latter purpose, the \imtaInlinecode{latex}{\subtitle} command helps define a subtitle as a part of the document's metadata, and %
is used inside of the \imtaInlinecode{latex}{\imtaMaketitlepage} command.


\subsubsection{\texttt{imtaMakeCover}}

This command outputs a cover as the last page of the document.
This page will always be a left page in a two-side document.

\section{External dependancies}

This package depends upon a number of external packages.
The use of these is explained hereafter, and the parameters each is used with are specified as well.
Furthermore, a code snippet is presented, that shows the import line and the settings of the corresponding package.

\subsection{\texttt{geometry}}

The \imtaInlinecode{latex}{geometry} package provides ways to act on the document's format.
This package defines a A4 format, with two-centimeter margins, and a top margin of an extra centimeter for the header.

\begin{imtaCode}{latex}
\RequirePackage[a4paper, margin=2cm, top=3cm]{geometry}
\end{imtaCode}

\subsection{\texttt{graphicx}}

The \imtaInlinecode{latex}{graphicx} package lets input graphics and pictures into the document.

\begin{imtaCode}{latex}
\RequirePackage{graphicx}
\end{imtaCode}

\subsection{\texttt{fontenc}}

The \imtaInlinecode{latex}{fontenc} package declares an encoding for the output font.
The \imtaInlinecode{bash}{imta_core} package uses a latin font whose encoding is \imtaInlinecode{latex}{T1}.

\begin{imtaCode}{latex}
\RequirePackage[T1]{fontenc}
\end{imtaCode}

\subsection{\texttt{hyperref}}

The \imtaInlinecode{latex}{hyperref} package helps typeset hypertext links.
The \imtaInlinecode{latex}{hidelinks} option hides the links, but keeps them clickable.
To output a hypertext link, use the \imtaInlinecode{latex}{\hyperref} command.

\begin{imtaCode}{latex}
\RequirePackage[hidelinks]{hyperref}
\end{imtaCode}

\subsection{\texttt{inputenc}}
\begin{imtaCode}{latex}
\RequirePackage[utf8]{inputenc}
\end{imtaCode}

\subsection{\texttt{fancyhdr}}
\begin{imtaCode}{latex}
\RequirePackage{fancyhdr}
\end{imtaCode}

\subsection{\texttt{tikz}}
\begin{imtaCode}{latex}
\RequirePackage{tikz}
\end{imtaCode}

\subsection{\texttt{titlesec}}
\begin{imtaCode}{latex}
\RequirePackage{titlesec}
\end{imtaCode}

\subsection{\texttt{titling}}
\begin{imtaCode}{latex}
\RequirePackage{titling}
\end{imtaCode}

\subsection{\texttt{sectsty}}
\begin{imtaCode}{latex}
\RequirePackage{sectsty}
\end{imtaCode}

\subsection{\texttt{etoolbox}}
\begin{imtaCode}{latex}
\RequirePackage{etoolbox}
\end{imtaCode}

\subsection{\texttt{hyphenat}}
\begin{imtaCode}{latex}
\RequirePackage[none]{hyphenat}
\end{imtaCode}

\subsection{\texttt{footmisc}}
\begin{imtaCode}{latex}
\RequirePackage[bottom]{footmisc}
\end{imtaCode}

 \chapter{Additional features: \texttt{imta\_extra}}

\end{document}
%%%%%%%%%% END %%%%%%%%%% 
%%%%%%%%%%%%%%%%%%%%%%%%% 
