%%%%%%%%%%%%%%%%%%%%%%%%%%%%%%%%%%%%%%%%%%%%%%%%%%%%%%%%%%%%%%%%%%%%%
% File Name:       imta_documentation
%
% Description:     documentation of the IMT Atlantique LaTeX Template.
%
% Note:            /
%
% Limitations:     /
%
% Errors:          None known
%
% Dependencies:    babel
%				   biblatex
%                  imta_core
%                  imta_extra
%
% Author:          A. Foucault - armand.foucault@telecom-bretagne.eu
% Contributors:    B. Porteboeuf - benoit.porteboeuf@telecom-bretagne.eu
%
% University :     IMT Atlantique, Brest (France)
%
% TeX Environment: TeXLive + pdfLaTeX
%%%%%%%%%%%%%%%%%%%%%%%%%%%%%%%%%%%%%%%%%%%%%%%%%%%%%%%%%%%%%%%%%%%%

\documentclass{report}


\usepackage{imta_core}
\usepackage{imta_extra}

\usepackage[english]{babel}

\usepackage{biblatex}
\addbibresource{imta_documentation_bibliography}


\author{FOUCAULT Armand\\PORTEBOEUF Benoît}
%\imtaAuthorShort{A. Foucault \& B. Porteboeuf}
\date{December 2017}
\title{Institut Mines-Télécom Atlantique - \LaTeX{} report template}
\subtitle{Documentation for the \texttt{imta} package}


\imtaSetIMTStyle

%%%%%%%%%%%%%%%%%%%%%%%%%%%%%%% 
%%%%%%%%%% BEGINNING %%%%%%%%%% 
\begin{document}

	
\imtaMaketitlepage

\tableofcontents

\newpage

\chapter{Core features: \texttt{imta\_core}}

\section{Introduction}
The \texttt{imta\_core} package provides a \LaTeX{} template that satisfies the IMT Atlantique corporate identity rules \cite{charteGraphique}. For slightly more advanced or specific features, you might want to refer to the \texttt{imta\_extra} package.

Since it is a package, it can be used for a variety of classes and geometry. It has been primarily designed to be usd in \texttt{article} and \texttt{report} documents, either with \texttt{oneside}/\texttt{twoside} or \texttt{onecolumn}/\texttt{twocolumn} geometries.

\section{Set up}
In order to use this template, you simply need to have a \TeX{} distribution installed, copy this file in your working directory and add the \imtaInlinecode{latex}{\usepackage{imta_core}} command in the preamble of your top document.

Note that this template is compatible with at least \TeX Live and MiK\TeX distributions, although we have noticed some issues might happen when trying to resolve dependencies with the latter. If so, then try to update your distribution using your Mik\TeX manager, as suggested in \cite{updatingMikTeX}.


\section{Sectioning}

\subsection{\texttt{imtaQuestion} and \texttt{imtaQuestionReset}}
The \imtaInlinecode{latex}{\imtaQuestion} command outputs and formats a question counter.
It's meant to be used in reports for assignment with questions.
The counter should be reset with the \imtaInlinecode{latex}{\imtaQuestionReset}.
This couple of commands is meant to be used for sectioning when the assignment does not use a more explicit titling.

\subsection{\texttt{subsubsubsection}}
The \imtaInlinecode{bash}{imta_core} package offers a one-level-deeper section than the usual deepest \imtaInlinecode{latex}{\subsubsection}.
This provides an alternative to the usual \imtaInlinecode{latex}{\paragraph}.

\subsection{\texttt{chapter}}

\section{Document styling}

\subsection{IMT Atlantique styling}
\subsubsection{Colors}
The core package defines four colors, including the three colors of the IMT Atlantique, and a uniform and arbitrary gray.
These are defined as follows:

\begin{imtaCode}{latex}
\definecolor{imtaGreen}{RGB}{164, 210, 51}
\definecolor{imtaLightBlue}{RGB}{0, 184, 222}
\definecolor{imtaDarkBlue}{RGB}{12, 35, 64}
\definecolor{imtaGray}{RGB}{87, 87, 87}
\end{imtaCode}

Here are samples of these colors, with text in both black and white for previsualising the contrast.

\begin{figure}[H]
    \centering
    \resizebox{10cm}{!}{%
    \begin{tikzpicture}
        \fill[color=imtaGray] (0, 9) rectangle (6, 11);
        \fill[color=imtaGray] (7, 9) rectangle (13, 11);
        \node at (3, 10) {\LARGE \bf imtaGray};
        \node[white] at (10, 10) {\LARGE \bf imtaGray};

        \fill[color=imtaDarkBlue] (0, 6) rectangle (6, 8);
        \fill[color=imtaDarkBlue] (7, 6) rectangle (13, 8);
        \node at (3, 7) {\LARGE \bf imtaDarkBlue};
        \node[white] at (10, 7) {\LARGE \bf imtaDarkBlue};

        \fill[color=imtaLightBlue] (0, 3) rectangle (6, 5);
        \fill[color=imtaLightBlue] (7, 3) rectangle (13, 5);
        \node at (3, 4) {\LARGE \bf imtaLightBlue};
        \node[white] at (10, 4) {\LARGE \bf imtaLightBlue};

        \fill[color=imtaGreen] (0, 0) rectangle (6, 2);
        \fill[color=imtaGreen] (7, 0) rectangle (13, 2);
        \node at (3, 1) {\LARGE \bf imtaGreen};
        \node[white] at (10, 1) {\LARGE \bf imtaGreen};
    \end{tikzpicture}
    }
    \caption{Samples of the IMT Atlantique colors}
    \label{fig:imtaColors}
\end{figure}

\subsubsection{\texttt{imtaSetIMTStyle}}
The official IMT Atlantique styling is not really \LaTeX-ish, and takes the decision to use a sans-serif font for body text.
Therefore, the default style uses the default \LaTeX{} font settings.
However, it is possible to enable a more IMT Atlantique-compliant styling, by calling the \imtaInlinecode{latex}{\imtaSetIMTStyle} command in the preamble.

The main aspects of the official style are:

\begin{itemize}
    \item Use of the Helvetica font for the body;
    \item Section titles in green (\imtaInlinecode{latex}{\imtaGreen}) and other heading titles in gray (\imtaInlinecode{latex}{\imtaGray});
    \item Section title in the header;
    \item Page number at the right corner of the footer.
\end{itemize}

For comparison, the default style of the template is:

\begin{itemize}
    \item Use of the default Computer Modern font for the body;
    \item Default style for headings: all in black;
    \item Document title at the left corner and author's name at the right corner of the header;
    \item Page number at the center of the footer.
\end{itemize}

\subsubsection{The IMT Atlantique logo with \texttt{imtaLogo} and \texttt{imtaLogoTikz}}
The IMT Atlantique logo can be output at the desired width with the \imtaInlinecode{latex}{\imtaLogo} command.
The latter includes an external pdf document, \imtaInlinecode{bash}{imta_logo.pdf}, that contains the official logo.
On the other hand, the \imtaInlinecode{latex}{\imtaLogoTikz} draws an approximation of the logo with the \imtaInlinecode{latex}{tikz} package.
The following is a comparison of both commands.

\begin{figure}[H]
    \centering
    \imtaLogo{5cm}
    \imtaLogoTikz{5cm}
    \caption{Comparison between \imtaInlinecode{latex}{imtaLogo} and \imtaInlinecode{latex}{imtaLogoTikz}}
    \label{fig:imtaLogo}
\end{figure}


\subsubsection{Front cover}
The \texttt{imtaMaketitlepage} command outputs a title page with the names of the authors, the date of writing, and the title of the document, along with the subtitle.
For this latter purpose, the \imtaInlinecode{latex}{\subtitle} command helps define a subtitle as a part of the document's metadata, and %
is used inside of the \imtaInlinecode{latex}{\imtaMaketitlepage} command.


\subsubsection{Back cover}

The \texttt{imtaMakeCover} command outputs a cover as the last page of the document.
This page will always be a left page in a two-side document.

\subsection{}

\section{External dependancies}

This package depends upon a number of external packages.
The use of these is explained hereafter, and the parameters each is used with are specified as well.
Furthermore, a code snippet is presented, that shows the import line and the settings of the corresponding package.

\subsection{\texttt{geometry}}

The \imtaInlinecode{latex}{geometry} package provides ways to act on the document's format.
This package defines a A4 format, with two-centimeter margins, and a top margin of an extra centimeter for the header.

\begin{imtaCode}{latex}
\RequirePackage[a4paper, margin=2cm, top=3cm]{geometry}
\end{imtaCode}

\subsection{\texttt{graphicx}}

The \imtaInlinecode{latex}{graphicx} package lets input graphics and pictures into the document.

\begin{imtaCode}{latex}
\RequirePackage{graphicx}
\end{imtaCode}

\subsection{\texttt{fontenc}}

The \imtaInlinecode{latex}{fontenc} package declares an encoding for the output font.
The \imtaInlinecode{bash}{imta_core} package uses a latin font whose encoding is \imtaInlinecode{latex}{T1}.

\begin{imtaCode}{latex}
\RequirePackage[T1]{fontenc}
\end{imtaCode}

\subsection{\texttt{hyperref}}

The \imtaInlinecode{latex}{hyperref} package helps typeset hypertext links.
The \imtaInlinecode{latex}{hidelinks} option hides the links, but keeps them clickable.
To output a hypertext link, use the \imtaInlinecode{latex}{\hyperref} command.

\begin{imtaCode}{latex}
\RequirePackage[hidelinks]{hyperref}
\end{imtaCode}

\subsection{\texttt{inputenc}}
\begin{imtaCode}{latex}
\RequirePackage[utf8]{inputenc}
\end{imtaCode}

\subsection{\texttt{fancyhdr}}
\begin{imtaCode}{latex}
\RequirePackage{fancyhdr}
\end{imtaCode}

\subsection{\texttt{tikz}}
\begin{imtaCode}{latex}
\RequirePackage{tikz}
\end{imtaCode}

\subsection{\texttt{titlesec}}
\begin{imtaCode}{latex}
\RequirePackage{titlesec}
\end{imtaCode}

\subsection{\texttt{titling}}
\begin{imtaCode}{latex}
\RequirePackage{titling}
\end{imtaCode}

\subsection{\texttt{sectsty}}
\begin{imtaCode}{latex}
\RequirePackage{sectsty}
\end{imtaCode}

\subsection{\texttt{etoolbox}}
\begin{imtaCode}{latex}
\RequirePackage{etoolbox}
\end{imtaCode}

\subsection{\texttt{hyphenat}}
\begin{imtaCode}{latex}
\RequirePackage[none]{hyphenat}
\end{imtaCode}

\subsection{\texttt{footmisc}}
\begin{imtaCode}{latex}
\RequirePackage[bottom]{footmisc}
\end{imtaCode}

\chapter{Additional features: \texttt{imta\_extra}}

\section{Introduction}
While the \texttt{imta\_core} package provides a \LaTeX{} template that satisfies the IMT Atlantique corporate identity rules, \cite{charteGraphique} the \texttt{imta\_extra} package provides the user some slightly more advanced or specific features.

Since it is a package, it can be used for a variety of classes and geometry. It has been primarily designed to be usd in \texttt{article} and \texttt{report} documents, either with \texttt{oneside}/\texttt{twoside} or \texttt{onecolumn}/\texttt{twocolumn} geometries.

\section{Set up}
In order to use this template, you need to have a \TeX{} distribution installed, copy this file as well as \texttt{imta\_core} in your working directory and add the \imtaInlinecode{latex}{\usepackage{imta_extra}} command in the preamble of your top document.

Since \texttt{imta\_extra} uses the \texttt{minted} package for code colouring, two extra steps are necessary to fully resolve dependencies. First, you need to have \texttt{Pygmentize} installed. If this is not the case, you can do it using Python utilitary \texttt{pip} by executing \imtaInlinecode{python}{pip install Pygments}. Finally, you also need to build your document with the correct options. If you are using pdf\LaTeX{} as your compiler, you need to add the \imtaInlinecode{latex}{--shell-escape} option.

Note that this template is compatible with at least \TeX Live and MiK\TeX distributions, although we have noticed some issues might happen when trying to resolve dependencies with the latter. If so, then try to update your distribution using your Mik\TeX manager, as suggested in \cite{updatingMikTeX}.

\section{Document Styling}
\subsection{Code Colouring}
\subsection{Structuring List of Figures and Tables}
\subsection{Page numbering}

\section{External dependancies}
\subsection{anyfontsize}
\subsection{imta\_core}
\subsection{mdframed}
\subsection{minted}


%\chapter{References}
\printbibliography[title=References,heading=bibintoc]

\end{document}

%%%%%%%%%% END %%%%%%%%%% 
%%%%%%%%%%%%%%%%%%%%%%%%% 
